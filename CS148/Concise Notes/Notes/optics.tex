\section{Optics}
\begin{defi}
	\textbf{Radiant Intensity} of a light source: $I(\omega)=\dv*{\Phi}{\omega}$
	\begin{itemize}
		\vspace{-0.5em}\item Total light power (exiting a light) per unit solid angle
		\vspace{-0.5em}\item Measure of how strong a (point) light source is
	\end{itemize}
\end{defi}
\begin{defi}
	\textbf{Irradiance} on a surface: $E=\dv*{\Phi}{A}$
	\begin{itemize}
		\vspace{-0.5em}\item Total light power (hitting a surface) per unit surface area.
		\vspace{-0.5em}\item Measure of how much light is hitting a surface.
		\vspace{-0.5em}\item Varies based on distance from the light and the tilting angle of the surface.
	\end{itemize}
\end{defi}
Some engineering approximations are as follows.
\begin{itemize}
	\item BRDF (Bidirectional Reflectance Distribution Function): models how much light is reflected.
	\item BTDF (Bidirectional Transmittance Distribution Function): models how much light is transmitted.
	\item BSSRDF (Bidirectional Surface Scattering Reflectance Distribution Function): combined reflection/transmission model.
\end{itemize}

Now we define the \textbf{lighting equation}:
\[
	L_o(\omega_0)=\sum_{i\in\mathrm{in}}L_{o\text{ due to }i}(\omega_i,\omega_o)
\]
where the BRDF gives each of $L_{o\text{ due to }i}(\omega_i,\omega_o)$.
Then we have
\[
	L_o(\omega_0)=\int_{i\in\mathrm{in}}\mathrm{BRDF}(\omega_i,\omega_0)\dd{E_i(\omega_i)}
	=
	\int_{i\in\mathrm{in}}\mathrm{BRDF}(\omega_i,\omega_0)L_i\cos\theta_i\dd{\omega_i}
.\] 

Diffuse Materials: a surface reflects light equally in all directions. I.e., $\mathrm{BRDF}=\mathrm{Const}$.
