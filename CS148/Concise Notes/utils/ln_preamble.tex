\usepackage{fancyhdr}
\pagestyle{fancy}
% \fancypagestyle{mainFancy}{
%     \fancyhf{}
%     \renewcommand\headrulewidth{.5pt}       % 页眉横线
%     \renewcommand\footrulewidth{0pt}
%     \fancyhead[OC]{\fzkai{\leftmark}}       % 页眉章标题
%     \fancyhead[EC]{\fzkai{\@title}}         % 页眉文章题目
% 	\lhead{\fzkai{author}}
%     \fancyhead[OR,EL]{\thepage}                 % 页眉编号
%     \fancyfoot[r]{\thumb} % 将拇指放到没有被使用的页眉或页脚处
% }
\fancyhead[L]{\slshape{Haoyu Zhen}}

% theorems
\usepackage{thmtools}
\usepackage{thm-restate}
\usepackage[framemethod=TikZ]{mdframed}
\mdfsetup{skipabove=1em,skipbelow=0em, innertopmargin=12pt, innerbottommargin=8pt}

\theoremstyle{definition}

\declaretheoremstyle[headfont=\bfseries\sffamily, bodyfont=\normalfont,
	mdframed={
		nobreak,
		backgroundcolor=brown!14,
		topline=false,
		rightline=false,
		leftline=true,
		bottomline=false,
		linewidth=2pt,
		linecolor=brown!180,
	}
]{thmbrownbox}

\declaretheoremstyle[headfont=\bfseries\sffamily, bodyfont=\normalfont,
	mdframed={
		nobreak,
		backgroundcolor=Blue!4,
		topline=false,
		rightline=false,
		leftline=true,
		bottomline=false,
		linewidth=2pt,
		linecolor=NavyBlue!120,
	}
]{thmbluebox}

\declaretheoremstyle[headfont=\bfseries\sffamily, bodyfont=\normalfont,
	mdframed={
		nobreak,
		backgroundcolor=Green!5,
		topline=false,
		rightline=false,
		leftline=true,
		bottomline=false,
		linewidth=2pt,
		linecolor=OliveGreen!120,
	}
]{thmgreenbox}

\declaretheoremstyle[headfont=\bfseries\sffamily, bodyfont=\normalfont,
	mdframed={
		nobreak,
		topline=false,
		rightline=false,
		leftline=true,
		bottomline=false,
		linewidth=2pt,
		linecolor=OliveGreen!120,
	}
]{thmgreenline}

\declaretheoremstyle[headfont=\bfseries\sffamily, bodyfont=\normalfont,
	mdframed={
		nobreak,
		topline=false,
		rightline=false,
		leftline=true,
		bottomline=false,
		linewidth=2pt,
		linecolor=NavyBlue!70,
	}
]{thmblueline}

\declaretheorem[numberwithin=section, style=thmbluebox, name={\color{Blue}Definition}]{defi}
\declaretheorem[numberwithin=section, style=thmgreenbox, name={\color{OliveGreen}Law}]{law}
\declaretheorem[numberwithin=section, style=thmbluebox, name={\color{Blue}Corollary}]{cor}
\declaretheorem[numberwithin=section, style=thmgreenline, name={\color{OliveGreen}Property}]{prt}
\declaretheorem[numberwithin=section, style=thmbluebox, name={\color{Blue}Proposition}]{prp}
\declaretheorem[numberwithin=section, style=thmbluebox, name={\color{Blue}Theorem}]{thm}
\declaretheorem[numberwithin=section, style=thmbluebox, name={\color{Blue}Lemma}]{lemma}
\declaretheorem[numberwithin=section, style=thmbrownbox,  name={\color{Brown}Example}]{eg}
\declaretheorem[numberwithin=section, style=thmgreenline, name={\color{OliveGreen}Remark}]{remark}
\declaretheorem[numbered=no,style=thmblueline, name={\color{NavyBlue!70}Proof},qed=$\square$]{prf}
\numberwithin{equation}{section}
