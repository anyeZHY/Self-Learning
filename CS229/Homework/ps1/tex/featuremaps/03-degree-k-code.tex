\item \subquestionpoints{5} {\bf Coding question: degree-$k$ polynomial regression}

Now we extend the idea above to degree-$k$ polynomials by considering $\phi:\mathbb{R}\rightarrow \mathbb{R}^{k+1}$ to be 
		\begin{align}
	\phi(x) = \left[\begin{array}{c} 1\\ x \\ x^2\\ \vdots \\x^k \end{array}\right]\in \mathbb{R}^{k+1} \label{eqn:feature-k}
	\end{align}

Follow the same procedure as the previous sub-question, and implement the algorithm with $k=3,5,10,20$. Create a similar plot as in the previous sub-question, and include the hypothesis curves for each value of $k$ with a different color. Include a legend in the plot to indicate which color is for which value of $k$.


Submit the plot in the writeup as the solution for this sub-problem. Observe how the fitting of the training dataset changes as $k$ increases. Briefly comment on your observations in the plot.


