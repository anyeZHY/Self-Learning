\section{Foundations}
\subsection{Model evaluation:}
Hold-out, cross validation and bootstrap.

For cross validation, we often let the numbers of the folds be 10. And in bootstrap, the equation $\lim_{n\to\infty}(1-\flatfrac{1}{m})^m=\flatfrac{1}{e}$ is used to analyse the probality.

\subsection{Performance}
\begin{defi}[Sensitivity and FPR]
	Now we consider that
	\begin{table}[H]
		\centering
		\begin{tabular}{c|c|c}
			       & prediction+ & prediction- \\ \hline
			Actual & 1           & 0           \\ \hline
			1      & TP          & FP          \\ \hline
			0      & FN          & TN
		\end{tabular}
	\end{table}
	\[
		TPR=\frac{TP}{TP+FN},\,
		FPR=\frac{FP}{TN+FP}
		.\]
\end{defi}
\begin{remark}
	{\textbf{ROC}} space and \href{https://baike.baidu.com/item/AUC/19282953}{\textbf{AUC}} is also useful to select models.

\end{remark}

\begin{defi}[Precision and recall]
	\[
		precision=\frac{TP}{TP+FP},\,
		recall=\frac{TP}{TP+FN}
		.\]
\end{defi}


\[
	F_\beta=\frac{(1+\beta^2)\times P\times R}{\beta^2\times P+R}
	.\]
$\beta$ depends on the preference of Precision and Recll.


\subsection{Bias-Variance Decomposition}
\begin{thm}
	\[
		\begin{aligned}
			E(f; D) & =
			bias^2(x)+var(x)+\varepsilon^2
			\\&=
			(\bar{f}(x)-y)^2+\mathbb{E}_D[f(x; D)-\bar{f}(x)]+\mathbb{E}_D[(y_D-y)^2]
		\end{aligned}
	\]

\end{thm}
