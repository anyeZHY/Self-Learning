\section{The Wave Function}

What we are looking for is the \textbf{wave function} $\*\Psi$.
\begin{law}[Schrodinger Equation]
	\[
		i\hbar\pdv{\Psi}{t}=-\frac{\hbar^2}{2m}\pdv[2]{\Psi}{x}+V\Psi
	.\]
\end{law}

For simplicity, we always rewrite it as:
\[
    i\hbar\partial_t\Psi=-\frac{\hbar^2}{2m}\partial_x^2\Psi+V\Psi
.\]

Born's statistical interpretation:
\[
	\int_{a}^{b} \abs{\Psi(x,t)}^2 \,\dd x=\text{probability of finding the particle between $a$ and $b$ at time $t$}
	.\]

\begin{law}[Normalization]
	\[
		\int_{-\infty}^{\infty} \abs{\Psi(x,t)}^2 \,\dd x = 1
		.\]
\end{law}

\begin{prp}
	The wave function will always stay NORMALIZED.
	\[
		\dv{t}\int_{-\infty}^{\infty} \abs{\Psi(x,t)}^2 \,\dd x = 0
		.\]
\end{prp}

\begin{prf}By Schrodinger EQ.,
	\[
		\LHS=\eval{\frac{i\hbar}{2m}\qty(\Psi^*\pdv{\Psi}{x}-\pdv{\Psi^*}{x}\Psi)}_{-\infty}^{+\infty}
		.\]
\end{prf}

\begin{defi}
	\[
		\angbr{x} \deq \int_{-\infty}^{\infty} x\abs{\Psi}^2 \,\dd x
	\]
	and
	\[
		\angbr{p} \deq m\dv{\angbr{x}}{t}
		.\]
\end{defi}

\begin{thm}
	\[
		\angbr{x}=\int \Psi^*(x)\Psi \,\dd x
	\]
	and
	\[
		\angbr{p}=\int \Psi^*\qty(-i\hbar\pdv{x})\Psi \,\dd x
		.\]
\end{thm}

\begin{remark}[Operator]
	We say that the operator $x$ represents position, and the operator $-i\hbar\pdv*{x}$ represents momentum. Also,
	\[
		\angbr{Q(x,p)}=\int_{-\infty}^{\infty} \Psi^*\qty[Q(x,-i\hbar\pdv{x})]\Psi \,\dd x
	.\] 
\end{remark}
\begin{prt}
	Operators do \textbf{NOT}, in general, commute. For example, $\hat{x}\hat{p}\ne \hat{p}\hat{x}$, i.e., 
	\[
		\exists\mbox{ a function }f,\ \mbox{s.t. }(\hat{x}\hat{p})f\ne (\hat{p}\hat{x})f
.\]
\end{prt}

\begin{thm}[de Broglie formula]
	The wave length is related to the momentum of the particle:
	\[
		p=\frac{h}{\lambda}=\frac{2\pi\hbar}{\lambda}
	.\] 
\end{thm}

\begin{thm}[Heisenberg's uncertainty principle]
	\[
	    \sigma_x\sigma_p\ge \frac{\hbar}{2}
	.\] 
\end{thm}


