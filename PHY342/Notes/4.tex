\section{Quantum Mechanics in Three Dimensions}
\subsection{The schrodinger Equation}
The generalization oto three dimensions is straitforward.
\[
	i\hbar\pdv{\Psi}{t}=-\frac{\hbar^2}{2m}\nabla^2\Psi+V\Psi
\] 
where
\[
	\nabla^2\equiv\pdv[2]{x}+\pdv[2]{y}+\pdv[2]{z}
\] 
is the \textbf{Laplacian}. Also the normalization conditions reads $\int\Psi\dd^3\*r=1$.
If  $V$ is independent of time, there will be a complete set of stationary states
 \[
	 \Psi_n(\*r,t)=\psi_n(\*r)e^{-\flatfrac{iE_nt}{\hbar}}
.\] 
Now we adopt spherical coordinates
\begin{lemma}[Laplacian in spherical coordinates]
	\[
		\nabla^2
		=
		\frac{1}{r^2}\pdv{r}\qty(r^2\pdv{r})
		+
		\frac{1}{r^2\sin\theta}\pdv{\theta}\qty(\sin\theta\pdv{\theta})
		+
		\frac{1}{r^2\sin^2\theta}\qty(\pdv[2]{\phi})
	.\] 
\end{lemma}

If $\Psi=R(r)Y(\theta,\phi)$ and $Y=\Theta(\theta)\Phi(\phi)$, we could separate $r,\theta$ and $\phi$ into three equations with important \textit{separation constants}.

\subsubsection{The angular Equation}
The $\phi$ equation is easy
 \[
	 \dv[2]{\Phi}{\phi}=-m^2\Phi \implies \Phi=e^{im\phi}
.\] 
When $\phi$ advances by  $2\pi$, we return to the same point in space, so it is natural to require that  $\Phi(\phi+2\pi)=\Phi(\phi)$. From this it follows that  $m$ must be an integer:
 \[
    m=0,\pm 1,\pm, 2,\cdots
.\] 
The $\theta$ equation reads
 \[
	 \sin\theta\dv{\theta}\qty(\sin\theta\dv{\Theta}{\theta})
	 +
	 \Big[l(l+1)\sin^2\theta-m^2\Big]\Theta
	 =
	 0
.\] 
\begin{lemma}[Legendre function]
	The solution of $\Theta$ is
	\[
	    \Theta(\theta)=AP^m_l(\cos\theta)
	.\] 
	where
	\[
	    P^m_l(x)
		\defeq
		(-1)^m(1-x^2)^{\flatfrac{m}{2}}\qty(\dv{x})^m P_l(x)
	\] 
	is the \textbf{associated Legendre function}, defined by the \textbf{Rodrigues formula}
	\[
	    P_l(x)
		\defeq
		\frac{1}{2^l l!}\qty(\dv{x})^l(x^2-1)^l
	.\] 
\end{lemma}
\begin{remark}
	Notice that $l$ must be a non-negative integer, for Rodrigues formula to make sense; moreover, if  $m>l$, we cwill have  $P^m_l(x)=0$. For any given  $l$, then there are  $2l+1$ possible values of  $m$:
	\[
		l=0,1,2\cdots \qand m=-l,\,-l+1,\cdots,l-1,\,l
	.\] 
\end{remark}

By normalization condition
\[
	\int_{0}^{\pi} \int_0^{2\pi} \abs{Y}^2\sin\theta \,\dd \theta\,\dd\phi=1
,\] 
we deduce that
\begin{equation}
    Y^m_l(\theta,\phi)
	=
	\sqrt{\frac{2l+1}{4\pi}\frac{(l-m)!}{(l+m)!}}e^{im\phi}P^m_l(\cos\theta)
	\label{yml}
\end{equation}
\subsubsection{The Radial Equation}
\begin{thm}[Radial equation]
	\[
		-\frac{\hbar^2}{2m}\dv[2]{u}{r}+\qty[V+\frac{\hbar^2}{2m}\frac{l(l+1)}{r^2}]u=Eu
	\] 
	where $u(r)\equiv rR(r)$.
\end{thm}
\begin{remark}[Effective potential]
	\[
		V_{\mathrm{eff}}=V+\frac{\hbar^2}{2m}\frac{l(l+1)}{r^2}
	\] and the latter term is the so-called \textbf{centrifugal potential}.
\end{remark}

\subsection{The Hydrogen Atom}
The radical equation says:
\[
    	-\frac{\hbar^2}{2m}\dv[2]{u}{r}
		+
		\qty[-\frac{e^2}{4\pi \varepsilon_0r}+\frac{\hbar^2}{2m}\frac{l(l+1)}{r^2}]u
		=
		Eu
.\] 
To tidy up the notation, let
\[
	\kappa=\frac{\sqrt{-2mE_e}}{\hbar},\ \ \ 
	\rho=\kappa r \qand
	\rho_0=\frac{m_e e^2}{2\pi\varepsilon_0\hbar^2\kappa}
\] 
so that
\[
	\dv[2]{u}{\rho}=\qty[1-\frac{\rho_0}{\rho}+\frac{l(l+1)}{\rho^2}]u
.\] 
Intuitively, ($\dv*[2]{u}{\rho}=u$ when $\rho\to+\infty$ and $\dv*[2]{u}{\rho}=\flatfrac{ul(l+1)}{\rho^2}$ when $\rho\to_0$)
\[
	u(\rho)=\rho^{l+1}e^{-\rho}v(\rho)
.\] 
Now we assume the solution, $v(\rho)$, can be expressed as a power series in  $\rho$:
 \[
	 v(\rho)=\sum_{j=0}^{+\infty}c_j\rho^j
.\] 
Plugin it into the radical equation
\[
	c_{j+1}=\qty{\frac{2(j+l+1)-\rho_0}{(j+1)(j+2l+2)}}c_j
.\] 
\begin{thm}
	The series must terminate. I.e., $\exists\, N\in \NN$, $c_N=0$, which means
	\[
	    2(N+l)-\rho_0=0
	.\] 
\end{thm}
\begin{prf}
	For large $j$, the recursion formula says
	\[
		c_{j+1}\approx \frac{2}{j+1}c_j
		\implies
		c_{j+1}\approx \frac{2^j}{j!}c_0
	.\] 
	Then
	\[
		v(\rho)=c_0e^{2\rho} \qand u(\rho)=c_0\rho^{l+1}e^{\rho}
	\] 
	which could not be \textbf{NORMALIZED}.
\end{prf}
\begin{thm}[Bohr Formula \& Radius]
	\[
		E_n=-\qty[\frac{m_e}{2\hbar^2}\qty(\frac{e^2}{4\pi\varepsilon_0})^2]
		\qand
		a=\frac{4\pi\varepsilon_0\hbar^2}{m_e e^2}
	.\]
\end{thm}

Finally, we obtain the spactial wave functions
\[
	\psi_{nlm}(r,\theta,\phi)=R_{nl}(r)Y^m_l(\theta,\phi)
\] 
where $R_{nl}(r)=r^{-1}\rho^{l+1}e^{-\rho}v(\rho)$ and $Y^m_l(\theta,\phi)$ is defined by Eq \ref{yml}. 
\begin{remark}[Laguerre Polynomials]
	\[
		v(\rho)=L^{2l+1}_{n-l-1}(2\rho)
	\] 
	where
	\[
		L^p_q(x)\defeq(-1)^p\qty(\dv{x})^pL_{p+q}(x)
	\] 
	is an associated Lguerre polynomial, and
	\[
		L_q(x)\defeq\frac{e^x}{q!}\qty(\dv{x})^q(e^{-x}x^q)
	\] 
	is the $q^{\text{th}}$ Laguerre polynomial. 
	\textbf{``Brutally''},
	\[
		\psi_{nlm}=\sqrt{\qty(\frac{2}{na})^3\frac{(n-l-1)!}{2n(n+l)!}}
		e^{-\flatfrac{r}{na}}\qty(\frac{2r}{na})^l
		\Big[L^{2l+1}_{n-l-1}(\flatfrac{2r}{na})\Big]Y^m_l(\theta,\phi)
	.\]
\end{remark}



\begin{defi}[Quantum Numbers]
	Intuitively,
	 \begin{itemize}
		\item 
			$n$ is the \textbf{principal quantum number}; it tells you the energy of electron.
		\item
			$l$ is called  \textbf{azimuthal quantum number} and $m$ the \textbf{megnetic quantum number}; they are related to the angular momentum of the electron.
	\end{itemize}
\end{defi}

\subsection{Angular Momentum}

