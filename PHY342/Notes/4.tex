\section{Quantum Mechanics in Three Dimensions}
\subsection{The schrodinger Equation}
The generalization oto three dimensions is straitforward.
\[
	i\hbar\pdv{\Psi}{t}=-\frac{\hbar^2}{2m}\nabla^2\Psi+V\Psi
\] 
where
\[
	\nabla^2\equiv\pdv[2]{x}+\pdv[2]{y}+\pdv[2]{z}
\] 
is the \textbf{Laplacian}. Also the normalization conditions reads $\int\Psi\dd^3\*r=1$.
If  $V$ is independent of time, there will be a complete set of stationary states
 \[
	 \Psi_n(\*r,t)=\psi_n(\*r)e^{-\flatfrac{iE_nt}{\hbar}}
.\] 
Now we adopt spherical coordinates
\begin{lemma}[Laplacian in spherical coordinates]
	\[
		\nabla^2
		=
		\frac{1}{r^2}\pdv{r}\qty(r^2\pdv{r})
		+
		\frac{1}{r^2\sin\theta}\pdv{\theta}\qty(\sin\theta\pdv{\theta})
		+
		\frac{1}{r^2\sin^2\theta}\qty(\pdv[2]{\phi})
	.\] 
\end{lemma}

If $\Psi=R(r)Y(\theta,\phi)$ and $Y=\Theta(\theta)\Phi(\phi)$, we could separate $r,\theta$ and $\phi$ into three equations with important \textit{separation constants}.

\subsubsection{The angular Equation}
The $\phi$ equation is easy
 \[
	 \dv[2]{\Phi}{\phi}=-m^2\Phi \implies \Phi=e^{im\phi}
.\] 
When $\phi$ advances by  $2\pi$, we return to the same point in space, so it is natural to require that  $\Phi(\phi+2\pi)=\Phi(\phi)$. From this it follows that  $m$ must be an integer:
 \[
    m=0,\pm 1,\pm, 2,\cdots
.\] 
The $\theta$ equation reads
 \[
	 \sin\theta\dv{\theta}\qty(\sin\theta\dv{\Theta}{\theta})
	 +
	 \Big[l(l+1)\sin^2\theta-m^2\Big]\Theta
	 =
	 0
.\] 
\begin{lemma}[Legendre function]
	The solution of $\Theta$ is
	\[
	    \Theta(\theta)=AP^m_l(\cos\theta)
	.\] 
	where
	\[
	    P^m_l(x)
		\defeq
		(-1)^m(1-x^2)^{\flatfrac{m}{2}}\qty(\dv{x})^m P_l(x)
	\] 
	is the \textbf{associated Legendre function}, defined by the \textbf{Rodrigues formula}
	\[
	    P_l(x)
		\defeq
		\frac{1}{2^l l!}\qty(\dv{x})^l(x^2-1)^l
	.\] 
\end{lemma}
\begin{remark}
	Notice that $l$ must be a non-negative integer, for Rodrigues formula to make sense; moreover, if  $m>l$, we cwill have  $P^m_l(x)=0$. For any given  $l$, then there are  $2l+1$ possible values of  $m$:
	\[
		l=0,1,2\cdots \qand m=-l,\,-l+1,\cdots,l-1,\,l
	.\] 
\end{remark}

By normalization condition
\[
	\int_{0}^{\pi} \int_0^{2\pi} \abs{Y}^2\sin\theta \,\dd \theta\,\dd\phi=1
,\] 
we deduce that
\[
    Y^m_l(\theta,\phi)
	=
	\sqrt{\frac{2l+1}{4\pi}\frac{(l-m)!}{(l+m)!}}e^{im\phi}P^m_l(\cos\theta)
.\] 
\subsubsection{The Radial Equation}
\begin{thm}[Radial equation]
	\[
		-\frac{\hbar^2}{2m}\dv[2]{u}{r}+\qty[V+\frac{\hbar^2}{2m}\frac{l(l+1)}{r^2}]u=Eu
	\] 
	where $u(r)\equiv rR(r)$.
\end{thm}
\begin{remark}[Effective potential]
	\[
		V_{\mathrm{eff}}=V+\frac{\hbar^2}{2m}\frac{l(l+1)}{r^2}
	\] and the latter term is the so-called \textbf{centrifugal potential}.
\end{remark}

\subsection{The Hydrogen Atom}
To be continued.

\subsection{Angular Momentum}

